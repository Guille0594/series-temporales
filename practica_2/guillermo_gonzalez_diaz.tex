% Options for packages loaded elsewhere
\PassOptionsToPackage{unicode}{hyperref}
\PassOptionsToPackage{hyphens}{url}
%
\documentclass[
]{article}
\usepackage{amsmath,amssymb}
\usepackage{lmodern}
\usepackage{iftex}
\ifPDFTeX
  \usepackage[T1]{fontenc}
  \usepackage[utf8]{inputenc}
  \usepackage{textcomp} % provide euro and other symbols
\else % if luatex or xetex
  \usepackage{unicode-math}
  \defaultfontfeatures{Scale=MatchLowercase}
  \defaultfontfeatures[\rmfamily]{Ligatures=TeX,Scale=1}
\fi
% Use upquote if available, for straight quotes in verbatim environments
\IfFileExists{upquote.sty}{\usepackage{upquote}}{}
\IfFileExists{microtype.sty}{% use microtype if available
  \usepackage[]{microtype}
  \UseMicrotypeSet[protrusion]{basicmath} % disable protrusion for tt fonts
}{}
\makeatletter
\@ifundefined{KOMAClassName}{% if non-KOMA class
  \IfFileExists{parskip.sty}{%
    \usepackage{parskip}
  }{% else
    \setlength{\parindent}{0pt}
    \setlength{\parskip}{6pt plus 2pt minus 1pt}}
}{% if KOMA class
  \KOMAoptions{parskip=half}}
\makeatother
\usepackage{xcolor}
\IfFileExists{xurl.sty}{\usepackage{xurl}}{} % add URL line breaks if available
\IfFileExists{bookmark.sty}{\usepackage{bookmark}}{\usepackage{hyperref}}
\hypersetup{
  pdftitle={PRÁCTICA 2: COLGATE VS CREST},
  pdfauthor={Guillermo González Díaz},
  hidelinks,
  pdfcreator={LaTeX via pandoc}}
\urlstyle{same} % disable monospaced font for URLs
\usepackage[margin=1in]{geometry}
\usepackage{graphicx}
\makeatletter
\def\maxwidth{\ifdim\Gin@nat@width>\linewidth\linewidth\else\Gin@nat@width\fi}
\def\maxheight{\ifdim\Gin@nat@height>\textheight\textheight\else\Gin@nat@height\fi}
\makeatother
% Scale images if necessary, so that they will not overflow the page
% margins by default, and it is still possible to overwrite the defaults
% using explicit options in \includegraphics[width, height, ...]{}
\setkeys{Gin}{width=\maxwidth,height=\maxheight,keepaspectratio}
% Set default figure placement to htbp
\makeatletter
\def\fps@figure{htbp}
\makeatother
\setlength{\emergencystretch}{3em} % prevent overfull lines
\providecommand{\tightlist}{%
  \setlength{\itemsep}{0pt}\setlength{\parskip}{0pt}}
\setcounter{secnumdepth}{-\maxdimen} % remove section numbering
\ifLuaTeX
  \usepackage{selnolig}  % disable illegal ligatures
\fi

\title{PRÁCTICA 2: COLGATE VS CREST}
\usepackage{etoolbox}
\makeatletter
\providecommand{\subtitle}[1]{% add subtitle to \maketitle
  \apptocmd{\@title}{\par {\large #1 \par}}{}{}
}
\makeatother
\subtitle{\emph{Predicción de series temporales}}
\author{Guillermo González Díaz}
\date{14/11/2021}

\begin{document}
\maketitle

{
\setcounter{tocdepth}{2}
\tableofcontents
}
\hypertarget{conclusiones-extrauxeddas-del-trabajo}{%
\subsubsection{\texorpdfstring{\textbf{Conclusiones extraídas del
trabajo:}}{Conclusiones extraídas del trabajo:}}\label{conclusiones-extrauxeddas-del-trabajo}}

\begin{itemize}
\tightlist
\item
  Los modelos AUTOARIMA se comportan bien con las series temporales
  estudiadas.
\item
  Nuestras series temporales tienen tendencia, estacionariedad y no
  presentan estacionalidad.
\item
  Hemos visto el poder que puede tener sobre el mercado los
  comunicaciones realizadas por organismos oficiales
\item
  Después de hallar la función de transferencia, los valores predictivos
  son mucho más acertados.
\end{itemize}

\hypertarget{gruxe1ficos-de-nuestras-series-temporales}{%
\section{\texorpdfstring{\emph{Gráficos de nuestras series
temporales}}{Gráficos de nuestras series temporales}}\label{gruxe1ficos-de-nuestras-series-temporales}}

Podemos observar varias cosas en estos gráficos. En primer lugar, ambas
series tienen tendencia. Crest creciente y Colgate decreciente. Podemos
ver que a partir de 1960, Crest supera en cuota de mercado a Colgate
para mantenerse de ese modo. En cuanto a la estacionalidad, no
observamos ninguna, ya que al tratarse de dentífricos sabemos que no son
porductos que presenten estacionalidad.

La explicación, es el informe publicado por el Consejo de Terapéutica
Dental de la American Dental Association (ADA), resaltando los
benificios de esta marca frente a otras. Así, vemos como en el período
seleccionado, Colgate pasó en solamente 5 años de una cuota de mercado
del 40 \% al 20 \%, en beneficio de Crest, una marca nueva que en solo 5
años alcanzó máximos del 50 \%.

\includegraphics{guillermo_gonzalez_diaz_files/figure-latex/unnamed-chunk-4-1.pdf}

\hypertarget{train-y-test}{%
\section{\texorpdfstring{\emph{Train y
Test}}{Train y Test}}\label{train-y-test}}

Hemos dividido la muestra en train y test, pues uno de los objetivos del
trabajo es predecir las últimas 16 semanas del período utilizando para
ello modelos ARIMA. Así queda partida la muestra:

\includegraphics{guillermo_gonzalez_diaz_files/figure-latex/unnamed-chunk-5-1.pdf}
\includegraphics{guillermo_gonzalez_diaz_files/figure-latex/unnamed-chunk-5-2.pdf}

\hypertarget{autorima-colgate-y-predicciuxf3n-16-peruxedodos}{%
\section{\texorpdfstring{\emph{Autorima Colgate y predicción 16
períodos}}{Autorima Colgate y predicción 16 períodos}}\label{autorima-colgate-y-predicciuxf3n-16-peruxedodos}}

En primer lugar realizamos un autoarima con las dos series.
Representamos gráficamete los residuos (en este caso solamente los de
Colgate, teniendo en cuenta que los de Crest se comportan de manera muy
similar), donde podemos comprobar que se comportan como ruido blanco: la
media es constante en torno a 0. Por ello podemos decir que los modelos
se ajustan bien y podemos predecir con ellos.

Estos son los resultados de las predicciones, en azul los intervalos de
confianza al 95 \%:

Análisis de los residuos de Colgate:
\includegraphics{guillermo_gonzalez_diaz_files/figure-latex/unnamed-chunk-8-1.pdf}

\includegraphics{guillermo_gonzalez_diaz_files/figure-latex/unnamed-chunk-9-1.pdf}

\hypertarget{autorima-crest-y-predicciuxf3n-16-peruxedodos}{%
\section{\texorpdfstring{\emph{Autorima Crest y predicción 16
períodos}}{Autorima Crest y predicción 16 períodos}}\label{autorima-crest-y-predicciuxf3n-16-peruxedodos}}

\includegraphics{guillermo_gonzalez_diaz_files/figure-latex/unnamed-chunk-11-1.pdf}

A continuación hemos comparado nuestras predicciones (en rojo) con los
valores reales. Vemos que nuestras predicciones, pese a ser lineas, se
ajustan bastante bien a lo que es la media de los datos reales, por lo
tanto podemos concluir que se trata de modelos fiables con buen
comportamiento.

\includegraphics{guillermo_gonzalez_diaz_files/figure-latex/unnamed-chunk-12-1.pdf}

\includegraphics{guillermo_gonzalez_diaz_files/figure-latex/unnamed-chunk-13-1.pdf}

\hypertarget{valores-atuxedpicos}{%
\section{\texorpdfstring{\emph{Valores
atípicos}}{Valores atípicos}}\label{valores-atuxedpicos}}

Colgate: observamos dos valores atípicos en las series. Uno en la semana
50 de 1959, y el otro en la 32 de 1960. El primero podemos definirlo
como un cambio temporal, puesto que pronto vuelve a su punto anterior y
tendencia alcista.

Sin embargo, el segundo es el que coincide con el informe mencionado
anteriormente, la semana 32 cayendo en agosto de 1960. Este podemos
decir que es un atípico aditivo, puesto que afecta a la serie temporal.

\includegraphics{guillermo_gonzalez_diaz_files/figure-latex/unnamed-chunk-14-1.pdf}

Crest: Aquí hemos encontrado 3 atípicos. Uno en la semana 32 de 1960,
que tiene el efecto contrario al mismo de Colgate. Otro en la semana 11
de 1961, que vemos que es puntual ya que se recupera rápidamente. Por
último, observamos uno en la semana 40 de 1961, donde la cuota car pero
se va recuperando poco a poco.

\includegraphics{guillermo_gonzalez_diaz_files/figure-latex/unnamed-chunk-15-1.pdf}

Llegados a este punto, podemos confirmar la fuerza que tuvo y el gran
impacto para la cuota de mercado de ambas marcas, la aprobación de Crest
como ``Categoría B'' por la ADA, confirmando la efectividad del
dentífrico ante las caries dentales.

Crest se mantendrá como rey del sector hasta los años 90.

\hypertarget{modelos-de-intervenciuxf3n}{%
\section{\texorpdfstring{\emph{Modelos de
intervención}}{Modelos de intervención}}\label{modelos-de-intervenciuxf3n}}

En el punto anterior nos hemos enconrtado con valores atípicos en ambos
casos. Como se trata de series temporales, no podemos limitarnos a
eliminarlos. Una serie temporal es continua y no podemos crear
discontinuidades quitando datos. Queremos ver el impacto que tienen
sobre la serie de datos, y principalmente son modelos que se realizan
con ARIMA.

Estos valores tiene sus causas en el mercado y las estrategias de
publicidad, y tienen su efecto en la tendencia de las series.

A continuación representamos los coeficientes de nuestro modelo de
intervención sobre ambas series temporales, y después pasamos a analizar
un outlier en concreto, el de agosto de 1960 y el efecto tan inmediato
que tuvo en Crest.

Modelo de intervención de colgate:

\begin{verbatim}
## 
## Call:
## arimax(x = int_data_col, order = c(3, 0, 0), seasonal = list(order = c(1, 0, 
##     0), period = 52), xreg = dummies_colg, method = "ML")
## 
## Coefficients:
##          ar1     ar2     ar3    sar1  intercept   AO1102   LS1136
##       0.2377  0.1615  0.1522  0.0307     0.3629  -0.1783  -0.0911
## s.e.  0.0619  0.0628  0.0626  0.0712     0.0087   0.0435   0.0122
## 
## sigma^2 estimated as 0.002052:  log likelihood = 435.5,  aic = -857
\end{verbatim}

Modelo de intervención de Crest:

\begin{verbatim}
## 
## Call:
## arimax(x = int_data_crest, order = c(3, 1, 0), seasonal = list(order = c(1, 
##     0, 0), period = 52), xreg = dummies_crest, method = "ML")
## 
## Coefficients:
##           ar1      ar2      ar3    sar1  LS1136   AO1167   TC1196
##       -0.6991  -0.4515  -0.1914  0.0650  0.1374  -0.1349  -0.0992
## s.e.   0.0621   0.0710   0.0623  0.0672  0.0323   0.0384   0.0320
## 
## sigma^2 estimated as 0.001768:  log likelihood = 452.81,  aic = -891.63
\end{verbatim}

Aquí podemos ver la gráfica con lo efectos del ya mencionado evento. En
crest podemos ver como tiene un efecto inmediato y positivo. Mirando
esta gráfica podemos comprobar que se trata de una intervención escalón.

\includegraphics{guillermo_gonzalez_diaz_files/figure-latex/unnamed-chunk-18-1.pdf}

\hypertarget{funciuxf3n-de-transferencia}{%
\section{\texorpdfstring{\emph{Función de
transferencia}}{Función de transferencia}}\label{funciuxf3n-de-transferencia}}

Por último. Para ello hemos modelizado con Arima, hemos comprobado el
número de coeficientes. Despúes con el análisis visual de los
coeficientes sacamos b, r y s. Con todo ello realizamos un nuevo modelo
Arimax, que usaremos para predecir las últimas 16 semanas de nuestra
serie tempral. Aquí tenemos el resultado, una predicción bastante más
ajustada que antes.

\includegraphics{guillermo_gonzalez_diaz_files/figure-latex/unnamed-chunk-19-1.pdf}

\end{document}
